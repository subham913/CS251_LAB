\documentclass[a4paper, 10pt]{report}


\usepackage{lipsum,lineno}
\usepackage{graphicx}
\usepackage{subcaption}
\usepackage{hyperref}


\usepackage{geometry}
 \geometry{
 a4paper,
 total={170mm,257mm},
 left=22mm,
 top=22mm,
 }


\title{Report}
\author{Subham Kumar}
\date{}
\begin{document}
\maketitle
 


\begin{figure}
\centering
\includegraphics[width=\columnwidth]{Threadscatter_1.eps}
 \caption{Scatter plot for Thread1}
 \label{fig:Threadscatter_1}
\end{figure}

This scatter plot is for number of threads equal to 1.This plot is between number of elements on x-axis and corresponding time of execution on y-axis.Time is in microseconds.
\newpage
\begin{figure}
\centering
\includegraphics[width=\columnwidth]{Threadscatter_2.eps}
 \caption{Scatter plot for Thread2}
 \label{fig:Threadscatter_2}
\end{figure}

This scatter plot is for number of threads equal to 2.This plot is between number of elements on x-axis and corresponding time of execution on y-axis.Time is in microseconds.
\newpage
\begin{figure}
\centering
\includegraphics[width=\columnwidth]{Threadscatter_4.eps}
 \caption{Scatter plot for Thread4}
 \label{fig:Threadscatter_4}
\end{figure}

This scatter plot is for number of threads equal to 4.This plot is between number of elements on x-axis and corresponding time of execution on y-axis.Time is in microseconds.
\newpage
\begin{figure}
\centering
\includegraphics[width=\columnwidth]{Threadscatter_8.eps}
 \caption{Scatter plot for Thread8}
 \label{fig:Threadscatter_8}
\end{figure}

This scatter plot is for number of threads equal to 8.This plot is between number of elements on x-axis and corresponding time of execution on y-axis.Time is in microseconds.
\newpage
\begin{figure}
\centering
\includegraphics[width=\columnwidth]{Threadscatter_16.eps}
 \caption{Scatter plot for Thread16}
 \label{fig:Threadscatter_16}
\end{figure}

This scatter plot is for number of threads equal to 16.This plot is between number of elements on x-axis and corresponding time of execution on y-axis.Time is in microseconds.






\newpage
\begin{figure}
\centering

\includegraphics[width=\columnwidth]{sl_1.eps}
 \caption{Line plot for Thread1}
 \label{fig:sl_1}
\end{figure}
This graph is a line plot for all the number of threads with number of elements on x-axis and average time for 100 sample on y-axis.Here time is in microseconds.
\newpage
\begin{figure}
\includegraphics[width=\columnwidth]{bar_plot.eps}
 \caption{This is Bar\_Plot}
 \label{fig:bar}
\end{figure}
This plot is a bar graph for Threads with number of elements on x-axis and average speedup time on y-axis.Here time is in microoseconds.
\newpage
\begin{figure}
\includegraphics[width=\columnwidth]{ebar_plot.eps}
 \caption{This is Error\_Bar\_Plot}
 \label{fig:error_bar}
\end{figure}
This plot is also a kind of bar plot with error-bars as additional feature.Here error bars have been calculated using variance.X-axis denotes number of elements and Y-axis denotes speedup time in microseconds. 



\end{document}

